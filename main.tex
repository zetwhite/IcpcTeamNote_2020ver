% Team Note Sample Template
% These codes should be guaranteed, fast enough, short and easy to type.

\documentclass[landscape, 10pt, a4paper, oneside, twocolumn]{extarticle}
\usepackage{kotex}
\usepackage{amssymb}
\usepackage{amsmath}
\usepackage{import}

\usepackage{teamnote}

\ShowUsage
\ShowComplexity
\HideAuthor

\begin{document}

\maketitlepage

% Make Pagebreak if you want.
% \pagebreak 


\section{General}

\Algorithm
{BFS}
{}
{}
{cpp}{source/BFS.cpp}
{roonm813}

\Algorithm
{DFS with recusiveCall}
{}
{}
{cpp}{source/DFS_recursiveCall.cpp}
{roonm813}

\Algorithm
{DFS with stack}
{}
{}
{cpp}{source/DFS_stack.cpp}
{roonm813}

\Algorithm
{Bellman Ford}
{}
{}
{cpp}{source/Bellman_Ford.cpp}
{gusah009}

\Algorithm
{Dijkstra}
{}
{}
{cpp}{source/Dijkstra.cpp}
{gusah009}

\Algorithm
{Floyd Warshall}
{}
{}
{cpp}{source/Floyd_Warshall.cpp}
{gusah009}

\Algorithm
{Ford Fulkerson}
{}
{}
{cpp}{source/Ford_Fulkerson.cpp}
{gusah009}

\Algorithm
{KMP}
{}
{}
{cpp}{source/KMP.cpp}
{roonm813}


\Algorithm
{Bipartite Matching}
{}
{}
{cpp}{source/BipartiteMatching.cpp}
{roonm813}

\Algorithm
{Knapsack}
{}
{}
{cpp}{source/Knapsack.cpp}
{roonm813}

\Algorithm
{MST}
{}
{}
{cpp}{source/MST.cpp}
{gusah009}

\section{Data Structure}

\Algorithm
{Disjoint Set}
{}
{}
{cpp}{source/DisjointSet.cpp}
{noonmap}

\Algorithm
{Segment Tree}
{}
{}
{cpp}{source/SegmentTree.cpp}
{noonmap}


\section{Number Theory}

\Algorithm
{Greatest Common Divider}
{}
{}
{cpp}{source/gcd.cpp}
{roonm813}


\Algorithm
{Extended Eculid}
{}
{}
{cpp}{source/ExtendedEuclid.cpp}
{roonm813}

\Algorithm
{Square and Multiply method}
{}
{}
{cpp}{source/Square_Multiply_Method.cpp}
{noonmap}

\section{Geometry}

\Algorithm
{ConvexHull}
{}
{}
{cpp}{source/ConvexHull.cpp}
{noonmap}

\Algorithm
{CrossLine Detection}
{}
{}
{cpp}{source/CrossLineDetection.cpp}
{noonmap}


\section{C++ STL examples}
\Algorithm
{SET}
{}
{}
{cpp}{source/Set.cpp}
{noonmap}

\Algorithm
{MAP}
{}
{}
{cpp}{source/Map.cpp}
{noonmap}

\section{Note}
\Algorithm
{tips}
{}
{}
{cpp}
{source/tips.cpp}
{roonm813}

\subsection{생각해보기}
``알고리즘 문제 해결 전략''에서 가져옴
\begin{itemize}
\item 비슷한 문제를 풀어본 적이 있던가?
\item 단순한 방법에서 시작할 수 있을까? (brute force)
\item 내가 문제를 푸는 과정을 수식화할 수 있을까? (예제를 직접 해결해보면서)
\item 문제를 단순화할 수 없을까?
\item 그림으로 그려볼 수 있을까?
\item 수식으로 표현할 수 있을까?
\item 문제를 분해할 수 있을까?
\item 뒤에서부터 생각해서 문제를 풀 수 있을까?
\item 순서를 강제할 수 있을까?
\item 특정 형태의 답만을 고려할 수 있을까? (정규화)
\end{itemize}

\end{document}